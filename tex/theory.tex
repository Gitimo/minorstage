\documentclass[a4paper]{article}

\usepackage[utf8]{inputenc}
\usepackage[T1]{fontenc}
\usepackage[english]{babel}
\usepackage[hmargin=3cm,vmargin=4cm]{geometry}
\usepackage{amsmath}
\usepackage{amssymb}
\usepackage{microtype}
\usepackage{color}
\usepackage{alltt}
\usepackage{upgreek}

\parindent0mm

\newcommand{\beq}{\begin{equation}}
\newcommand{\eeq}{\end{equation}}
\begin{document}

\section{Theory}

\subsection{Light absorption and carrier creation in GaAs solar cells}


\subsection{Carrier recombination in GaAs solar cells}
Once an electron-hole pair has been created by the absorption of a photon, there are two major possibilities for its development: either the pair is successfully separated by the junction, traverses the external circuit and is active in providing energy, or the electron-hole pair can recombine. Recombination can either be radiative or non-radiative. In radiative recombination, a photon of corresponding energy is released, whereas in non-radiative recombination, either a phonon is released (Shockley-Read-Hall recombination) or the excess energy is passed on to a third carrier (Auger recombination).
\begin{itemize}
\item Charge collection
\item Recombination
	\begin{itemize}
	\item Radiative Recombination
	\item Non-Radiative Recombination
		\begin{itemize}
		\item Shockley-Read-Hall recombination
		\item Auger recombination
		\end{itemize}
	\end{itemize}
\end{itemize}
In solar cells, non-radiative recombination is an unwanted phenomenon since it constitutes loss of generated carriers and thus loss of energy. Auger recombination is a minor issue: it is a three body collision process and thus intrinsically improbable to happen. Non-radiative recombination at trap-states within the energy-gap is, however, still a major concern. Such trap-states can be created either by impurities or imperfections in the crystal lattice. With current high-quality GaAs growth techniques, impurities are of less concern than imperfections, especially when one realises that every crystal surface constitutes such an imperfection. The `ideal', `perfect' crystal is a purely theoretical construct, it's translation symmetry is never broken, it is infinite. To avoid loss of energy through recombination at the back side of the solar cell, a back surface field is induced in the material. It is an electrical field which `pushes' the carriers away from the surfaces, back into the bulk.

\subsection{Rau's reciprocity relation}
In 2007, Rau published a paper with interesting reciprocity relations relating electrical properties of a solar cell to optical properties of the same device operated as a LED. It will become clear that from a simple, standardised measurement of the external-quantum-efficiency (\emph{EQE}) of a solar cell, a wealth of information can be gained. A mathematically rigorous derivation is provided by Rau, relying on a more fundamental theorem by Donolato. However, in the following, we will chose an intuitive approach to obtain the same results because they implications of the formul\ae thus derived will become clear more easily. The interested reader is referred to the original publication for the more mathematical approach.\\

We start with Planck's radiation law in its most well-known form:
\beq
B_{\lambda} (T) = \int _{\lambda_1} ^{\lambda 2} \frac{2  \,  \pi  \, h  \, c^2}{\lambda ^5} \frac{d\lambda}{\exp(\frac{hc}{\lambda kT})-1} \, ,
\eeq
where $h$ is Planck's constant, $c$ the speed of light, $k$ the Boltzmann constant and $T$ the absolute temperature. $B_{\lambda} (T)$ gives the energy density of a black-body at temperature $T$ in terms of the wavelength, but we are more interested in the corresponding flux density: Using
\beq
\label{enlam}
E(\lambda)=\frac{hc}{\lambda}
\eeq
the spectral photon density becomes
\beq
\label{photlam}
\Phi_{\lambda} (T) = \int _{\lambda_1} ^{\lambda 2} \frac{2  \,  \pi  \, c}{\lambda ^4} \frac{d\lambda}{\exp(\frac{hc}{\lambda kT})-1} \, .
\eeq
The derivative of \eqref{enlam} is:
\beq
\frac{dE}{d\lambda}=-\frac{hc}{\lambda^2} \, .
\eeq
So:
\begin{align}
\Phi_{\lambda} (T) &= \int _{\lambda _1} ^{\lambda _2} \frac{2  \,  \pi  \, c}{\lambda ^4} \frac{d\lambda}{\exp(\frac{hc}{\lambda kT})-1} \\
\Phi_{E} (T)	&=\int _{E(\lambda _2)} ^{E(\lambda_1)} \frac{2 \, \pi }{h \, \lambda ^2} \frac{dE}{\exp(\frac{hc}{\lambda kT})-1} \\
		&=\int _{E(\lambda _2)} ^{E(\lambda_1)} \frac{2  \,  \pi}{h^3 \, c^2} \frac{E ^2 \, dE}{\exp(\frac{E}{kT})-1} \, ,
\end{align}
which means that in thermal equilibrium, each surface element of a solar cell is irradiated from each element of the spherical angle $\theta$ of the ambient with a spectral flux density:
\begin{align}
\phi _{eq} (E_{\lambda},T,\theta)&=\phi_{E_{\lambda}}(T)cos(\theta)\\
				&=\frac{2 }{h^3 \, c^2} \frac{E_{\lambda} ^2 \, \cos (\theta)}{\exp(\frac{E_{\lambda}}{kT})-1} \\
				&\approx \frac{2 E_{\lambda} ^2 \, \cos (\theta)}{h^3 \, c^2} \exp (\frac{-E_{\lambda}}{kT}) \, ,
\end{align}
where $E_{\lambda}$ is the energy of photons of wavelength $\lambda$. The last approximation can be used to obtain analytical solutions if all spectra are taken as black-body radiation functions. However, since it is more common in the field of solar cells to use the AM1.5 reference spectrum, we will use the exact formulation and obtain numerical results.\\
A portion $\alpha (r_S, \theta, \upvarphi, E_{\lambda})$ of that radiation will be absorbed by the solar cell, depending on the coordinate $r_S$ of the surface element $dS$ of the cell and the angles of incidence $\theta$ \& $\upvarphi$. Accordingly, the probability that one photon impinging on the cell will contribute one elementary charge $q$ to the short-circuit current of the cell will be given by $Q(r_S, \theta, \upvarphi, E_{\lambda})=\alpha F_C$, where $F_C$ is the probability for a charge carrier to be collected by the junction~\footnote{in the field of photo-voltaics, the quantity $Q$ is known as the external-quantum-efficiency of a solar cell}. Thus, from thermal radiation, there will be a short-circuit current component
\begin{equation}
J_{SC,0}=q \int _{\Omega _{c}} \int _{E_{\lambda}} \int _{S_C} Q(r_S, \theta, \upvarphi, E_{\lambda}) \phi_{eq} (\theta, E_{\lambda}) d\Omega dE_{\lambda}dS \, ,
\end{equation}
which will be called the dark-equilibrium-short-circuit current. The integral extends over the full spherical angle $\Omega _{c} = 2\pi$. 
Because of thermal equilibrium, no net current will be flowing, so we can postulate that $J_{SC,0}$ will be counterbalanced by an equilibrium injection current $J_{em,0}=J_{SC,0}$ for which the relation 
\begin{equation}
\updelta J_{em,0}=\updelta J_{SC,0} = q Q \phi_{eq}
\end{equation}
also holds throughout the entire cell. Thus, it can be seen that in thermal equilibrium, light injection with \emph{subsequent carrier collection} is precisely counter-balanced by current injection with \emph{subsequent light emission}.\\
However, under illumination from the sun, an excess flux density $\phi_{sun}$ impinges on the cell, so the short-circuit current will be given by
\begin{equation}
J_{SC}=q \int _{\Omega _{sun}} \int _{E_{\lambda}} \int _{S_C} Q(r_S, \theta, \upvarphi, E_{\lambda}) \phi_{sun} (\theta , \upvarphi , E_{\lambda}) d\Omega dE_{\lambda}dS \, .
\end{equation}
Rau proceeded to postulate that the injection current $\updelta J_{em}$ under an applied bias voltage would follow an exponential law and that it were superimposed on the dark-equilibrium-short-circuit current, so that the excess photon flux density $\phi_{em}$ emitted from the device would follow Shockley's diode law:
\begin{equation}
\phi_{em}(r_S, \theta, \upvarphi, E_{\lambda})=Q(r_S, \theta, \upvarphi, E_{\lambda}) \phi_{eq} (\theta, E_{\lambda}) (\exp (\frac{qV}{kT})-1) \, .
\end{equation}
Thus, from a known $Q$ -- i.e. \emph{EQE} -- the spectral and angular emmittance of the device if operated as a LED can be calculated. The derivation is strictly valid only for angular \emph{EQE} measurements, but Green has shown that deviations from the exact results are small enough to allow an approximation of the method by measuring just the perpendicular component of the \emph{EQE}.\\
Furthermore, through the relation $V_{OC}=kT/q \ln (J_{SC}/J_0 +1)$ we can calculate a maximum attainable open-circuit voltage for a given device, once we realise that the minimum recombination current will be given by the equilibrium injection current:
\begin{equation}
\label{eq:vocrad}
V_{OC}^{rad}=\frac{kT}{q} \ln (\frac{J_{SC}}{J_{em,0}}+1) \, .
\end{equation}
This maximum open circuit voltage is in a sense more realistic than Shockley and Queisser's limit in that it takes into account the devices actual quantum efficiency and does not model the absorption as a step function $\alpha(E_{\lambda})=1$ for $E_{\lambda}>E_g$ and  $\alpha(E_{\lambda})=0$  for $E<E_g$. Equation \eqref{eq:vocrad} can be seen as the most important result of this derivation, since it will allow us to compare the performance of solar cells based on their maximum attainable open-circuit voltage, \emph{the} parameter which has to be improved upon to achieve another world-record solar cell at \emph{AMS}.

\end{document}
