\documentclass[a4paper]{article}
\usepackage[english]{babel}
\usepackage[hmargin=3cm,vmargin=4cm]{geometry}
\usepackage{amsmath}
\usepackage{amssymb}
\usepackage{microtype}
\usepackage{color}
\usepackage{alltt}
\pagestyle{empty}
\newcommand{\beq}{\begin{equation}}
\newcommand{\eeq}{\end{equation}}
\begin{document}
We start with Planck's radiation law in its most well-known form:
\beq
B_{\lambda} (T) = \int _{\lambda_1} ^{\lambda 2} \frac{2  \,  \pi  \, h  \, c^2}{\lambda ^5} \frac{d\lambda}{\exp(\frac{hc}{\lambda kT})-1} \, ,
\eeq
where $h$ is Planck's constant, $c$ the speed of light, $k$ the Boltzmann constant and $T$ the absolute temperature. $B_{\lambda} (T)$ gives the energy density of a black-body at temperature $T$ in terms of the wavelength, but we are more interested in the corresponding flux density: Using
\beq
\label{enlam}
E(\lambda)=\frac{hc}{\lambda}
\eeq
the spectral photon density becomes
\beq
\label{photlam}
P_{\lambda} (T) = \int _{\lambda_1} ^{\lambda 2} \frac{2  \,  \pi  \, c}{\lambda ^4} \frac{d\lambda}{\exp(\frac{hc}{\lambda kT})-1} \, .
\eeq
The derivative of \eqref{enlam} is:
\beq
\frac{dE}{d\lambda}=-\frac{hc}{\lambda^2} \, .
\eeq
So:
\begin{align}
P_{\lambda} (T) &= \int _{\lambda _1} ^{\lambda _2} \frac{2  \,  \pi  \, c}{\lambda ^4} \frac{d\lambda}{\exp(\frac{hc}{\lambda kT})-1} \\
P_{E} (T)	&=\int _{E(\lambda _2)} ^{E(\lambda_1)} \frac{2 \, \pi }{h \, \lambda ^2} \frac{dE}{\exp(\frac{hc}{\lambda kT})-1} \\
		&=\int _{E(\lambda _2)} ^{E(\lambda_1)} \frac{2  \,  \pi}{h^3 \, c^2} \frac{E ^2 \, dE}{\exp(\frac{E}{kT})-1}
\end{align}





\end{document}
